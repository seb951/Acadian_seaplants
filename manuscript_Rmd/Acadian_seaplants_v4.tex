\documentclass[12pt,]{article}
\usepackage[left=1in,top=1in,right=1in,bottom=1in]{geometry}
\newcommand*{\authorfont}{\fontfamily{phv}\selectfont}
\usepackage[]{mathpazo}


  \usepackage[T1]{fontenc}
  \usepackage[utf8]{inputenc}



\usepackage{abstract}
\renewcommand{\abstractname}{}    % clear the title
\renewcommand{\absnamepos}{empty} % originally center

\renewenvironment{abstract}
 {{%
    \setlength{\leftmargin}{0mm}
    \setlength{\rightmargin}{\leftmargin}%
  }%
  \relax}
 {\endlist}

\makeatletter
\def\@maketitle{%
  \newpage
%  \null
%  \vskip 2em%
%  \begin{center}%
  \let \footnote \thanks
    {\fontsize{18}{20}\selectfont\raggedright  \setlength{\parindent}{0pt} \@title \par}%
}
%\fi
\makeatother




\setcounter{secnumdepth}{0}



\title{\textbf{Seaweed extracts strongly structured microbial communities
associated with tomato and pepper roots and significantly increased crop
yield}  }



\author{\Large Sébastien Renaut\textsuperscript{1,2},Jacynthe
Masse\textsuperscript{1,2}, Jeffrey P. Norrie\textsuperscript{3}, Bachar
Blal\textsuperscript{3} Mohamed Hijri\textsuperscript{1,2}\vspace{0.05in} \newline\normalsize\emph{\textsuperscript{1}Département de Sciences Biologiques, Institut de
Recherche en Biologie Végétale, Université de Montréal, 4101 Sherbrooke
Est, Montreal, H1X 2B2, Quebec, Canada. \textsuperscript{2}Quebec Centre
for Biodiversity Science, Montreal, Quebec, Canada
\textsuperscript{3}Acadian Seaplant Ltd, 30 Brown Avenue, Darthmouth,
Nova Scotia, Canada, B3B 1X8}  }


\date{}

\usepackage{titlesec}

\titleformat*{\section}{\normalsize\bfseries}
\titleformat*{\subsection}{\normalsize\itshape}
\titleformat*{\subsubsection}{\normalsize\itshape}
\titleformat*{\paragraph}{\normalsize\itshape}
\titleformat*{\subparagraph}{\normalsize\itshape}





\newtheorem{hypothesis}{Hypothesis}
\usepackage{setspace}

\makeatletter
\@ifpackageloaded{hyperref}{}{%
\ifxetex
  \PassOptionsToPackage{hyphens}{url}\usepackage[setpagesize=false, % page size defined by xetex
              unicode=false, % unicode breaks when used with xetex
              xetex]{hyperref}
\else
  \PassOptionsToPackage{hyphens}{url}\usepackage[unicode=true]{hyperref}
\fi
}

\@ifpackageloaded{color}{
    \PassOptionsToPackage{usenames,dvipsnames}{color}
}{%
    \usepackage[usenames,dvipsnames]{color}
}
\makeatother
\hypersetup{breaklinks=true,
            bookmarks=true,
            pdfauthor={Sébastien Renaut\textsuperscript{1,2},Jacynthe
Masse\textsuperscript{1,2}, Jeffrey P. Norrie\textsuperscript{3}, Bachar
Blal\textsuperscript{3} Mohamed Hijri\textsuperscript{1,2} (\textsuperscript{1}Département de Sciences Biologiques, Institut de
Recherche en Biologie Végétale, Université de Montréal, 4101 Sherbrooke
Est, Montreal, H1X 2B2, Quebec, Canada. \textsuperscript{2}Quebec Centre
for Biodiversity Science, Montreal, Quebec, Canada
\textsuperscript{3}Acadian Seaplant Ltd, 30 Brown Avenue, Darthmouth,
Nova Scotia, Canada, B3B 1X8)},
             pdfkeywords = {Stella Maris®, 16S, ITS, soil microbial diversity, Illumina MiSeq, ANE,
Amplicon Sequence Variants, OTU},  
            pdftitle={\textbf{Seaweed extracts strongly structured microbial communities
associated with tomato and pepper roots and significantly increased crop
yield}},
            colorlinks=true,
            citecolor=blue,
            urlcolor=blue,
            linkcolor=magenta,
            pdfborder={0 0 0}}
\urlstyle{same}  % don't use monospace font for urls

% set default figure placement to htbp
\makeatletter
\def\fps@figure{htbp}
\makeatother

\usepackage[left]{lineno}
\linenumbers


% add tightlist ----------
\providecommand{\tightlist}{%
\setlength{\itemsep}{0pt}\setlength{\parskip}{0pt}}

\begin{document}
	
% \pagenumbering{arabic}% resets `page` counter to 1 
%
% \maketitle

{% \usefont{T1}{pnc}{m}{n}
\setlength{\parindent}{0pt}
\thispagestyle{plain}
{\fontsize{18}{20}\selectfont\raggedright 
\maketitle  % title \par  

}

{
   \vskip 13.5pt\relax \normalsize\fontsize{11}{12} 
\textbf{\authorfont Sébastien Renaut\textsuperscript{1,2},Jacynthe
Masse\textsuperscript{1,2}, Jeffrey P. Norrie\textsuperscript{3}, Bachar
Blal\textsuperscript{3} Mohamed Hijri\textsuperscript{1,2}} \hskip 15pt \emph{\small \textsuperscript{1}Département de Sciences Biologiques, Institut de
Recherche en Biologie Végétale, Université de Montréal, 4101 Sherbrooke
Est, Montreal, H1X 2B2, Quebec, Canada. \textsuperscript{2}Quebec Centre
for Biodiversity Science, Montreal, Quebec, Canada
\textsuperscript{3}Acadian Seaplant Ltd, 30 Brown Avenue, Darthmouth,
Nova Scotia, Canada, B3B 1X8}   

}

}








\begin{abstract}

    \hbox{\vrule height .2pt width 39.14pc}

    \vskip 8.5pt % \small 

\noindent Seaweeds and their derivatives have been used as a source of natural
fertilizer and biostimulant in agriculture and horticulture for
centuries. However, their effects on soil and crop roots microbiota
remain unclear. Here, we used a commercially available \emph{Ascophyllum
nodosum} extract in order to test its effect on bacterial and fungal
communities of rhizospheric soils and roots of pepper and tomato plants
in greenhouse trials. Two independent greenhouse trials were conducted
using tomato and pepper plants grown in natural soil in a split block
design with four treatments (planted, non-planted, fertilized and
non-fertilized). We used amplicon sequencing targeting fungal ITS and
bacterial 16S rRNA gene to determine microbial community structure
changes in the rhizosphere soil and root biotopes. We find that all
productivity measures of root, shoot and fruit biomass differed
significantly according to crop species, and most of those were
significantly greater according to the fertilization treatment. In
addition, \(a\)-diversity differed according to fertilization, but this
effect was small. Species composition among sites (\(b\)-diversity)
differed according to fertilization in all four communities measured
(fungal-root, fungal-soil, bacterial-root and bacterial-soil). Finally,
we identified a number of candidate taxa most strongly correlated with
crop yield increases. Further studies on isolation and characterization
of these microbial taxa that are linked to the application of liquid
seaweed extract may help to enhance crop yield and sustain
agro-ecosystems.


\vskip 8.5pt \noindent \emph{Keywords}: Stella Maris®, 16S, ITS, soil microbial diversity, Illumina MiSeq, ANE,
Amplicon Sequence Variants, OTU \par

    \hbox{\vrule height .2pt width 39.14pc}



\end{abstract}


\vskip 6.5pt


\noindent \doublespacing \newpage 

\section{INTRODUCTION}\label{introduction}

Seaweeds (also known as marine macroalgae) have been used as a source of
organic matter and mineral nutrients for centuries, especially in
coastal areas (Khan et al., 2009; Craigie, 2011). Liquid seaweed
extracts, developed in the 1950s in order to concentrate plant
growth-stimulating compounds, facilitate their usage (Milton, 1952).
Today, most commercially available extracts are made from the brown
algae \emph{Ascophyllum nodosum}, \emph{Ecklonia maxima} or
\emph{Laminaria spp}. Unlike modern chemical fertilizers, seaweed
extracts are biodegradable, non-toxic and come from a renewable resource
(Dhargalkar \& Pereira, 2005). Industry-funded bodies such as the
European Biostimulant Industry Coalition and the United States
Biostimulant Coalition have been working to accommodate biostimulants
into mainstream legal architecture. These organizations extoll benefits
arising from modes-of-action research, agricultural applications and
positive effects on yield and quality of many commercial species
(i.e.~fruits, vegetables, turf, ornamentals and woody species). Legal
recognition will further allow a fluid integration of various
biostimulants, including \emph{Ascophyllum nodosum} Extracts (ANE) into
sustainable long-term crop management programs (Craigie, 2011; Jardin,
2015).\\
\hspace*{0.333em}\\
Several comprehensive reviews have described the effects of seaweed
extracts on agricultural plant productivity (Khan et al., 2009; Craigie,
2010, 2011; Battacharyya et al., 2015). The science points to
wide-ranging effects from biotic to abiotic resistance, effects on
growth and development, and ultimately, to their impact on plant
establishment, crop yield and/or quality, and shelf life. At the
physiological level, these extracts have been found to influence
hormonal changes that in turn, influence physiological processes even at
very low concentrations (Wally et al., 2013).\\
\hspace*{0.333em}\\
Starting in the 1990's, high quality ANE was developped and let to an
increased usage by farmers, in addition to an increase in cause-effect
research, especially on plant diseases (Jayaraj \& Ali, 2015). Noted
increases in the activity of superoxide dismutase, glutathione
peroxidase and ascorbate peroxidase helped support the argument that ANE
improve plant tolerance to oxidative stress (Ayad et al., 1997; Schmidt
\& Zhang, 1997; Ayad, 1998; Allen et al., 2001). Positive effects were
also found on phytoalexin production suggesting that ANE may be involved
in suppressing disease infection through increased activity of these
protective enzymes that target oxidizing toxins naturally emitted by
disease pathogens (Lizzi et al., 1998; Jayaraj et al., 2008; Jayaraman,
Norrie \& Punja, 2010).\\
\hspace*{0.333em}\\
Improved plant stress resistance and tolerance to foliar and soil
treatments is attributed to a cascade of various physiological
reactions. ANE can impact plant-signalling mechanisms through a
multitude of plant processes and cellular modifications including
osmotic/oxidative stresses such as salinity, freezing and drought stress
(Jithesh et al., 2012). ANE can also impart drought-stress tolerance to
plants by reducing stomatal conductance and cellular electrolyte leakage
(Shotton and Martynenko, unpublished data; Spann \& Little, 2011). These
results indicate that ANE can influence cellular membrane maintenance
leading to a higher tolerance for various osmotic stresses and can
mitigate oxidative damage.\\
\hspace*{0.333em}\\
Although there is an abundance of published evidence detailing systemic
plant effects from ANE, outstanding questions remain as to the effects
of ANE on the rhizosphere biology. Various microbes, small arthropods,
nematodes and insects thrive in the soil rhizosphere. This microbial
biodiversity then contributes to the aggregation of soil particles,
enhances nutrient cycling and delivery to plants, degrades toxic
substances, allows better soil water retention and plays a role in plant
disease management. It has been suggested that the plant immune system
is composed of inherent surveillance systems that perceive several
general microbial elicitors, which allow plants to switch from growth
and development into a defense mode (Newman et al., 2013). This may
allow the plant to avoid infection from potentially harmful microbes. An
examination of sustainable products that can positively influence
microbial interactions between plant roots and soil biota will in turn
help to further understand soil borne plant-pathogens competition
dynamics. The effect of ANE on the bacterial profile suggests that ANE
applications increased strawberry root and shoot growth, berry yield and
rhizosphere microbial diversity and physiological activity (Alam et al.,
2013). Similar results were found in sandy loam soils as Alam and
colleagues Alam et al. (2014) showed a strong relationship between
carrot growth, soil microbial populations and activity.\\
\hspace*{0.333em}\\
The recent development of culture-independent molecular techniques and
high throughput sequencing should permit to circumvent the inherent
biases of culture-based approaches by targeting the ubiquitous component
of life, DNA. In turn, this will help to identify a larger proportion of
the microbial diversity and lead to a better understanding of the soil
microbial response to seaweed extract. DNA barcoding targeting specific
regions of the genome (e.g.~ITS: fungi, 16s ribosomal genes: bacteria)
are now regarded as a prerequisite procedure to comprehensively document
the diversity and ecology of microbial organisms (Toju et al., 2012;
Klindworth et al., 2013).\\
\hspace*{0.333em}\\
Here the general objective was to quantify the impact of ANE on plant
growth and test how the bacterial and fungal communities responded to
the addition of these extracts. We also aimed to identify specific taxon
positively correlated with increases in plant productivity following ANE
amendments. We hypothesized that the addition of liquid seaweed extracts
would improve productivity and alter significantly the bacterial and
fungal communities. We used a commercially available ANE, Stella Maris®,
developed by Acadian Seaplants Ltd (NS, Canada). Stella Maris® is
derived from the marine algae \emph{A. nodosum}, and harvested from the
nutrient-laden waters of the North Atlantic off the Eastern Coast of
Canada. We tested the effect of ANE on two agricultural plants commonly
grown in greenhouse conditions (tomato and pepper). Several traits
related to plant productivity were measured and soil and root bacterial
and fungal diversity were quantified using High Throughput Illumina (San
Diego, CA, USA) Miseq sequencing.\\
\hspace*{0.333em}\\
\newpage  

\section{MATERIAL AND METHOD}\label{material-and-method}

\emph{Experimental design}\\
Greenhouse trials were set up in large trays (60x30x18 cm LxWxH) using
two different crops: tomato (\emph{Solanum lycopersicum} L.) and pepper
(\emph{Capsicum annuum} L.). Tomato cultivar Totem Hybrid\#A371 was
planted in November 16th 2015 and pepper cultivar Ace Hybrid\#318 was
planted in December 9th 2015. Tomato and pepper seeds were purchased
from William Dam Seeds Ltd (ON, Canada). These cultivars were selected
for greenhouse production. Soil was collected from an agricultural field
under organic regime at the IRDA research station in St-Bruno (Qc,
Canada, 45\textsuperscript{o}32'59.6``N,
73\textsuperscript{o}21'08.0''W) on October 7th 2015. The soil was a
loamy sand and was collected from the 15 cm top layer. Natural soil was
mixed and put into trays, filled to 15 cm in height. Soil analysis was
done using a commercial service provided by Environex (formerly
AgriDirect, Longueuil, QC) and soil characteristics are shown in Table
S1. Eight seeds per tray were planted and after germination, only four
seedlings per tray were kept.\\
\hspace*{0.333em}\\
For each crop species, a randomized split block design (Table S2) was
used with four trays set up per block and eight blocks for each trial.
Half of the trays were fertilized (fertilization treatment), as
described below. Half of the trays were also planted (planting
treatment) with four plants per tray, while the other trays were not
planted. This allowed a direct comparison of fungal and bacteria soil
communities with respect to fertilization and planting treatments.\\
\hspace*{0.333em}\\
Two different fertilization regimes were used according to the plant
species. For tomatoes, plants were fertilized using multipurpose organic
fertilizer (pure hen manure, 18 g per tray repeated every 4 weeks,
5-3-2) from Acti-sol (Notre-Dame-du-Bon-Conseil, QC) in addition to
Stella Maris® (3.5 ml per 1L, each tray received 250 ml, repeated every
2 weeks) for the duration of the experiment. The other half was not
fertilized, but watered with 250ml per tray instead. The
physico-chemical composition of Stella Maris® is shown in Table S3. For
the pepper experiment, the fertilization regime consisted solely of
Stella Maris® (3.5 ml per 1L, each tray received 250 ml, repeated every
2 weeks) for the duration of the experiment. The other half was not
fertilized but watered with 250 ml per tray instead. Both experiments
were managed under organic farming practices. Thrips were controlled
using \emph{Neoseiulus cucumeris} (syn. \emph{Amblyseius cucumeris}) (1
bag per plant), Fungus gnats were also controlled using predatory mite
\emph{Gaeolaelaps gillespiei} (1L; Natural Insect Control, ON). Plants
were treated once a week with Milstop, a Potassium Bicarbonate-based
foliar fungicide to control the powdery mildew on both crops.\\
\hspace*{0.333em}\\
\emph{Plant productivity}\\
Tomato and pepper experiments were harvested on March 29th 2016. The
following traits assessed plant productivity: fruit number, fruit
weight, shoots fresh weight and roots fresh weight. Traits were measured
on three plants chosen randomly per tray for each fertilization /
control treatment, crop (tomato / pepper) and block (eight blocks) for a
total of 96 samples. In addition, both shoot and root samples were dried
in a 70 degrees drying oven, and dry weights were quantified after 48
hours. Together, these traits are expected to represent well the plant
overall productivity.\\
\hspace*{0.333em}\\
\emph{Sample preparation, DNA extraction and High throughput
sequencing}\\
Soil and root samples were taken for both experiments. Soil DNA was
extracted using NucleoSpin® Soil DNA extraction kit (Macherey-Nagel,
BioLinx, ON) on 250 mg of soil, following the manufacturer's protocol.
Roots were first washed with tap water and rinsed with sterile water.
Chopped roots sub-samples (100 mg) were subjected to DNA extraction
using DNeasy Plant Mini kit (Qiagen Inc - Canada, ON), following the
manufacturer's recommendations. Amplicon sequencing targeting bacterial
16S rRNA gene and fungal ITS was performed on both root and soil
samples.\\
\hspace*{0.333em}\\
For fungal ITS, we used the following primers with the universal CS1 and
CS2 adapters: CS1\_ITS3\_KYO2 (5'-ACA CTGA CGA CAT GGT TCT ACA GAT GAA
GAA CGY AGY RAA-3') and CS2\_ITS4\_KYO3 (5'-TAC GGT AGC AGA GAC TTG GTC
TCT BTT VCC KCT TCA CTC G-3') to produce a final amplicon size of
approximately 430bp including adapters (Toju et al., 2012).\\
\hspace*{0.333em}\\
For bacterial 16S, we used the following primers with CS1 and CS2
universal adapters: 341F (5'-CCT ACG GGN GGC WGC AG-3') and 805R (5'-GAC
TACC AGG GTA TCT AAT C-3') to produce a final amplicon size of
approximately 460 bp and targeting specifically the bacterial V3-V4
region of the 16S ribosomal gene (Klindworth et al., 2013).\\
\hspace*{0.333em}\\
DNA samples were then barcoded, pooled and sequenced (2X300bp,
paired-end) using an Illumina MiSeq sequencer through a commercial
service provided by the Genome Quebec Innovation Centre (Montreal, QC).
Sequences were demultiplexed by the sequencing facility and further
processed as described below.\\
\hspace*{0.333em}\\
\emph{Bioinformatics}\\
All bioinformatics, statistical, and graphical analyses further
described were performed in R 3.5.1 (R Core Team, 2018) and detailed
scripts are available here
(\url{https://github.com/seb951/Acadian_Seaplants}).\\
\hspace*{0.333em}\\
We used the \texttt{R} package \texttt{DADA2} (Callahan et al., 2016) to
infer \emph{Amplicon Sequence Variants} (ASV). \texttt{DADA2} offers
accurate sample inference from amplicon data with single-nucleotide
resolution in an open source environment. Unlike the Operational
Taxonomic Unit (OTU) approach (e.g. Schloss et al., 2009; Caporaso et
al., 2010), ASV are not treated as cluster of sequences defined with an
\emph{ad hoc} sequence similarity threshold. Instead, after sequences
are quality trimmed and error-corrected, \texttt{DADA2} reveals the
unique members of the sequenced community, thus allowing sequences and
abundance counts to be comparable among studies (Callahan et al.,
2016).\\
\hspace*{0.333em}\\
First, sequences were trimmed following strict quality thresholds
(removing primers and low quality nucleotides, see parameter details in
the accompanying \texttt{R} scripts). Following this, we applied the
error model algorithm of \texttt{DADA2}, which incorporates quality
information after filtering, unlike other OTU based methods. Then
dereplication, sample inference, merging of paired end reads and removal
of chimera were performed in order to obtain a sequence (ASV) table of
abundance per sample. Taxonomy was assigned through the \texttt{DADA2}
pipeline using the Ribosomal Database Project (RDP) Naive Bayesian
Classifier algorithm from Wang \emph{et al.} (2007). Depending on
support (minimum bootstrap support of 80), we assigned taxonomy from
Kingdom to species. We used the silva database formatted for
\texttt{DADA2} to infer bacterial taxa (Callahan, 2018). We used the
Unite (Community, 2018) fasta release (including singletons) to infer
fungal taxa after formatting it to the \texttt{DADA2} format using a
custom \texttt{R} script. The pipeline was run on a multithreaded (48
CPUs) computer infrastructure provided by Westgrid
(\url{https://www.westgrid.ca/support/systems/cedar}) and Compute Canada
(www.computecanada.ca). Note that the pipeline was run separately for
fungal-root, fungal-soil, bacteria-soil and bacteria-root samples given
the markedly different nucleotide compositions of the sequenced
amplicons, unique taxa and specific error models of each dataset. ~\\
\hspace*{0.333em}\\
\emph{Statistical analyses - plant productivity}\\
We tested for the effect of species (tomato vs pepper), fertilization
and their interaction on six plant productivity measures (fruit number,
average fruit weight, shoots fresh weight, roots fresh weight, shoots
dry weight, roots dry weight). We used Linear Mixed effect Models (LMM)
in the R package \texttt{NLME} (Pinheiro et al., 2017), which are more
appropriate than an Analysis of Variance (ANOVA) given the current block
design (blocks and replicates were treated as random variables). All six
plant productivity measures were either square root or log transformed
in order to help satisfy the assumption of normality of the residuals in
the LMM statistical framework. For the variables \emph{fruit number} and
\emph{average fruit weight}, we also verified statistical significance
using a permutation-based 2-way ANOVA (Anderson \& Legendre, 1999) given
that the residuals of the LMM were not normally distributed. Results
were similar according to the 2-way ANOVA.\\
\hspace*{0.333em}\\
\emph{Statistical analyses - microbial and fungal diversity}\\
Fungal-root, fungal-soil, bacterial-root and bacterial-soil ASV
diversity was measured separately. For each of these four datasets, we
removed samples that showed poor sequencing output and contained few
ASV. In order to do this, we summed the abundance of all ASV for each
sample (\(\sum_{i=1}^n ASV\)) and eliminated samples that had fewer that
the mean sum minus four standard deviations
(\(\overline{\sum_{i=1}^n ASV} - 4\sigma\)). In addition, we removed ASV
from our dataset that were present in fewer than 5\% of the samples
(less than ten individuals in the soil samples or less than five in the
root samples). This was done to remove very rare ASV unique to a block
or replicate, but not found in the majority of samples.\\
\hspace*{0.333em}\\
We then conducted community-based analyses looking at the effect of the
fertilization treatment on ASV abundance in the tomato and pepper
experiments. To reduce the complexity of the datasets, relative
abundance of all taxa was calculated per family using the \texttt{R}
package \texttt{DPLYR} (Wickham et al., 2015). Barplots were drawn using
\texttt{GGPLOT2} (Wickham, 2016) to visualize communities. ASV alpha
(\(a\))-diversity was calculated based on all ASV (excluding rare ASV,
see paragraph above) for each sample using the inverse Simpson diversity
index in \texttt{VEGAN} (Oksanen et al., 2013). The effect of the
fertilization treatment, species (and planting for soil communities)
were assessed using a Linear Mixed effect Model (LMM) model in the R
package \texttt{NLME} (Pinheiro et al., 2017), given the unbalanced,
replicated block design. Alpha diversity was \emph{log} transformed in
order to help satisfy the assumption of normality of the residuals in
the LMM statistical framework. ~\\
Using the community matrix data of ASV abundance, we performed
PERmutational Multivariate ANalysis Of VAriance tests (PERMANOVA;
Anderson, 2001) to identify relationships between the communities
according to the experimental design. ASV abundance matrix was
Hellinger-transformed and significance was assessed using 10,000
permutations in \texttt{vegan} (Oksanen et al., 2013). Blocks and
replicates were factored as strata in the model.\\
\hspace*{0.333em}\\
We also performed canonical correspondence analyses (CCAs) using the
Hellinger-transformed ASV abundance matrix in \texttt{vegan} (Oksanen et
al., 2013) to visually assess the grouping of samples, ASV and their
association with productivity variables (\emph{species} scaling based on
ASV matrix). Data were analyzed separately for fungal-root, fungal-soil,
bacterial-root and bacterial-soil, but also according to species
(tomato/pepper), given that analyses of \(a\)-diversity showed that
tomato and pepper were markedly different. This gave a total of eight
CCAs. Data were constrained based on four productivity measures (fruit
number, average fruits weight, shoots fresh weight, roots fresh weight).
We excluded the shoots \& roots dry weights as constraints to simplify
the model. In addition, these were highly correlated with the fresh
weight already included as constraints (\(r^2\)=0.98 and 0.76 for shoot
dry/fresh weights and root dry/fresh weights, respectively).\\
\hspace*{0.333em}\\
Finally, we attempted to identify candidate ASV positively associated
with productivity. As such, we identified the ten ASV most positively
associated with the measures of fruit number, shoots fresh weight and
roots fresh weight from each canonical correspondence analysis for a
total of 40 fungal and 40 bacterial candidate ASV. We aligned candidate
sequences from these candidates ASV using the Bioconductor \texttt{R}
package \texttt{DECIPHER} (Wright, 2016) and build pairwise distances
matrices using a JC69 substitution models of DNA sequence evolution
(equal base frequencies, Jukes \& Cantor, 1969) in \texttt{PHANGORN}
(Schliep, 2010). Phylogenetic trees (neighbour-joining) for bacteria and
fungi were plotted using \texttt{APE} (Paradis, Claude \& Strimmer,
2004). This permitted to identify if similar candidate ASV were found
under different experimental conditions (soil/root, pepper/tomato), thus
reinforcing their role in productivity increase and decreasing the false
positive rate.\\
\hspace*{0.333em}

\newpage  

\section{RESULTS}\label{results}

The effects of the fertilization treatment were determined by measuring
six agronomic parameters (fruit number, average fruit weight, shoots
fresh weight, shoots dry weight, roots fresh weight, roots dry weight)
for both tomatoes and peppers. We observed a significant increase of all
these agronomic parameters for fertilized plants except for the average
fruit fresh weight for tomato that did not differ between fertilized and
control plants (LMM, \(F_{(1,69)}\) = 1.27, \emph{p}-value=0.26, Figure
1 and Figure S1). The fertilization effect was stronger in the tomato
plants (fold changes between fertilized and control plants shown in
Figure 1), likely due to the fact that these plants were fertilized with
both hen manure and ANE. In addition, the model revealed a significant
interaction between treatment and plant (\(F_{(1,69)}\) = 9.6,
\emph{p}-value=0.0028). In fact, when testing only the pepper plants,
the effect of fertilization on average fruit weight was significantly
higher in the fertilized pepper plants (\(F_{(1,23)}\) = 10.84,
\emph{p}-value=0.0032).\\
\hspace*{0.333em}\\
\emph{Amplicon Sequencing}\\
A total of 2.7 million paired-end raw reads were obtained for all
samples combined (976,000 for fungi-soil, 920,000 for fungi-root,
309,000 for bacteria-soil and 535,000 for bacteria-root, Table S4). We
analyzed separately the sequence datasets for fungal-soil, fungal-root,
bacteria-soil and bacteria-root conditions. On average, 47,664
paired-end reads were obtained per sample. After quality filters were
applied, including removing chimeras, and paired-end reads were merged,
an average of 19,690 sequences remained per sample. From 192 soil
samples for fungi and bacteria, and 96 root samples for fungi and
bacteria sequenced, seven fungi-soil samples, 15 fungi-root samples and
one bacteria-root samples were removed because they had to few reads
based on our strict quality thresholds.\\
\hspace*{0.333em}\\
The \texttt{DADA2} pipeline inferred, on average, 170 Amplicon Sequence
Variants (ASV) per sample (average of 176 fungal-soil ASV, 37
fungal-root ASV, 269 bacterial-soil ASV and 92 bacterial-root ASV). Many
of these were unique to one or a few samples (total number of 6,112
fungal-soil, 845 fungal-root, 9,352 bacterial-soil and 2,023
bacterial-roots ASV). After quality filtering, we retained 413, 106, 811
and 325 ASV respectively for fungal-soil, fungal-root, bacterial-soil
and bacterial-roots. These retained ASV comprised 94\%, 95\%, 89\% and
98\% of all reads in the fungal-soil, fungal-root, bacterial-soil and
bacterial-root samples, respectively.\\
\hspace*{0.333em}\\
\emph{Fungal and bacterial diversity in root and soil biotopes}\\
The microbial community structures of soil and root samples were
analyzed and the relative abundance of their taxa was determined at the
family level (Figures 2 \& 3). Fungal communities were dominated by
Nectriaceae, both in the root and soil samples, while the bacterial
family Bacilaceae dominated to a lesser extent the soil samples.
Bacterial root communities were largely dominated by Cyanobacteria
(identified as \emph{chloroplast} in the silva database according to the
RDP Bayesian Classifier).\\
\hspace*{0.333em}\\
\emph{Local (\(a\)-diversity)}\\
The \(a\)-diversity of each biotope (soil or root) was calculated
separately for each sample and under each experimental condition
(fungi-soil, fungi-root, bacteria-soil and bacteria-root, Figure 4).
Linear mixed effects models showed that the \(a\)-diversity was
significantly higher in the soil biotope that in the roots for both
fungi and bacteria.\\
\hspace*{0.333em}\\
In soil samples, fungal \(a\)-diversity was significantly different in
planted compared to non-planted treatments (\(F_{(1,161)}\)=9.0,
\emph{p}-value=0.0032) and tomato versus pepper (\(F_{(1,161)}\)=13.03,
\emph{p}-value=0.0003), while no significant change was observed in
fertilized versus non-fertilized treatments (\(F_{(1,161)}\)=0.17,
\emph{p}-value=0.6853). In root samples, fungal \(a\)-diversity was
significantly different in fertilized versus non-fertilized treatments
(\(F_{(1,56)}\)=10.1, \emph{p}-value=0.003) and tomato versus pepper
(\(F_{(1,56)}\)=4.5, \emph{p}-value=0.04). ~\\
In soil samples, bacterial \(a\)-diversity was significantly different
in fertilized versus non-fertilized treatments (\(F_{(1,165)}\)=17.13,
\emph{p}-value\textless{}0.0001), in planted compared to non-planted
treatments (\(F_{(1,165)}\)=139.0, \emph{p}-value\textless{}0.0001),
while no significant change was observed in tomato versus pepper
(\(F_{(1,165)}\)=1.89, \emph{p}-value=0.17). In root samples,
bacterial\(a\)-diversity was significantly different in fertilized
versus non-fertilized treatments (\(F_{(1,67)}\)=17.27,
\emph{p}-value=0.0001), and tomato versus pepper (\(F_{(1,67)}\)=359.69,
\emph{p}-value\textless{}0.0001).\\
\hspace*{0.333em}\\
\emph{Differences in species composition among sites}\\
Using a PERMANOVA, the fertilization treatment had a highly significant
effect on both fungal and bacterial community structures (Table 1).
Planting also had a significant effect (greatest \% of variance was
explained by the planted factor) on fungal and bacterial community
structures. Plant identity (tomato/pepper) significantly influenced the
fungal and bacterial community structures in roots. ~\\
Canonical correspondence analyses (CCAs, Figures 5 for fungi and Figure
6 for bacteria) illustrated that roots fresh weight, shoots fresh weight
and fruit number responded similarly, while average fruit weight behaved
differentially as noted previously in (in fact nearly orthogonally to
the other three parameters in most ordinations). In addition, it showed
that fertilized samples clustered together and were positively
correlated with increases in these four productivity measures.\\
\hspace*{0.333em}\\
Next, we identified, for each ordination, the ten ASV most closely
related to the three constraints of the model (roots fresh weight,
shoots fresh weight and fruit number). These ASV were considered as
putative candidate taxa most positively impacted by increases in
productivity due to the fertilization treatment. We further analyzed the
corresponding sequences for these eighty candidate ASV (ten candidates *
eight ordinations) in two separate alignments (one for fungi and one for
bacterial ASV) and their accompanying phylogenetic trees.\\
\hspace*{0.333em}\\
In fungi, we identified one cluster of ASV taxonomically assigned to
\emph{Mortierella} (soil saprotrophs in the phylum Mucoromycota)
positively associated to productivity in both tomato and pepper roots
(Figure S2). In addition, we identified a cluster of four different
fungal ASV in tomato soil (ASV132, ASV153) and pepper-root (ASV19 \&
ASV17) closely related phylogenetically. Given that no taxonomy was
assigned to these sequences through the \texttt{DADA2} RDP bootstrap
approach, we used a BLASTn (Altschul et al., 1997) approach to identify
the most closely related sequences against NCBI nr. However, the most
closely related reference sequences were from uncultured fungus clones
(BLASTn, 88\% identity, e-value=1e-55). The remaining ASV were
identified as several different species such as \emph{Fusarium sp.},
\emph{Microdochium colombiense} or \emph{Setophoma terrestris}, known as
endophytes or pathogens.\\
\hspace*{0.333em}\\
In bacteria-roots, we identified a large diversity of ASV positively
correlated (increased abundance of these ASV) with the four measures of
productivity. Phylogenetic analyses did not reveal clusters of ASV
associated with increases in productivity in the four different
experimental conditions (Figure S3). \newpage  

\section{DISCUSSION}\label{discussion}

In the current study, we investigated the effects of \emph{Ascophyllum
nodosum} extracts (ANE) on root, shoot and fruit biomass in addition to
identifying bacterial and fungal communities in tomato and pepper.
Overall parameters related to plant growth (root, shoot and fruit
weights) significantly increased in both plant species in response to
ANE application. These results corroborate previous studies documenting
the impact of ANE on productivity in strawberries (Alam et al., 2013)
and carrots (Alam et al., 2014).\\
\hspace*{0.333em}\\
In the tomato experimental set up, the effect of fertilization was
especially high, likely due to the fact that plants were also fertilized
with hen manure in addition to ANE (see Figure 1). This was not the case
for the pepper plants and the increase in productivity was solely due to
the addition of ANE. The commercial extract used in this investigation
contained about 0.1\% nitrogen, 0.2\% phosphorus, 5\% potassium, along
with several micronutrients (Table S3) and it is sold as a complement to
fertilizers because it contains all microelements required for plant
growth. In the current experimental setup, ANE was diluted to 3.5 ml/L
prior to application (250 ml per tray every two weeks). In fact, in the
tomato plants the amounts of N and P supplied via the application of ANE
were 200-1000 times less than from the hen manure itself. As such, these
nutrients were given at very low concentrations relative to the crop
requirements and are not expected to significantly impact growth
relative to a regular agricultural fertility program ({\textbf{???}}).
Instead, organic molecules such as betaines, polyamines, cytokinins,
auxins, oligosaccharides, amino acids and vitamins present in ANE have
been found to have overall beneficial productivity effects on plant
growth (Khan et al., 2009; Craigie, 2010, 2011; Battacharyya et al.,
2015).\\
\hspace*{0.333em}\\
One of primary goal of the study was to document how bacterial and
fungal communities responded to the addition of ANE. We used a
metabarcoding high throughput sequencing approach targeting DNA regions
specific to fungi (ITS) and bacteria (16S). Then, we identified
bacterial and fungal taxa present in the samples using a relatively
novel bioinformatics approach developed by Callahan et al. (2016). The
approach, based on the widely used programming language \texttt{R} (R
Core Team, 2018), identifies unique, non-clustered sequences (ASV) that
are then comparable among studies. In addition, the current analytical
pipeline uses a bayesian classifier for taxonomy rather than the widely
used BLAST approach, thus providing more conservative, but more accurate
taxonomic identifications (Wang et al., 2007).\\
\hspace*{0.333em}\\
In the current experimental set up for both plants, most ASV identified
were rare and unique to one or a few sample. In fact, approximately 90\%
of all ASV were discarded given that they were found in singletons or
present in very few samples and were thus not representative of a
particular experimental treatment. These `rare' ASV comprised a small
minority of all sequencing reads (approximately 5\% of all sequences), a
pattern reminiscent of the early species abundance models showing that
in most ecological communities, few species are exceptionally abundant
whereas most are rare (Fisher, Corbet \& Williams, 1943).\\
\hspace*{0.333em}\\
The fertilization treatment had a significant effect on both fungal and
bacterial \(a\)-diversity (total number of ASV) in the root biotope. In
the soil biotope, it only had a significant effect for bacteria (Figure
4). Nectriaceae, a family of fungi in the order Hypocreales and often
encountered as saprotrophes on decaying organic matter comprised most of
the diversity both in the soil and plant roots (between 25-70\% of the
total number of sequencing reads, Figure 2). With respect to bacterial
communities of the soil, theses were much more diverse and comprised
many different families (Figure 3). Surprisingly, most sequencing reads
in the bacterial communities of roots likely originate from the plants
themselves (identified as chloroplastic or mitochondrial in origin in
Figure 3), despite the fact that the DNA primers pair used should have
primarily targeted the bacterial V3-V4 region of the 16S ribosomal
gene.\\
\hspace*{0.333em}\\
Fertilization treatment significantly influenced fungal and bacterial
community composition (\(b\)-diversity) among root and soil biotopes.
This fertilization effect was small (2-7\% of variance explained in the
models, Table 1) but significant, implying that the adddition of ANE
(pepper) or ANE and hen manure (tomato) has a small impact on microbial
communities. In fact, most of the variance in soil communities was
explained by the planting effect, showing how plants can alter their
microbiome. In the root biotope, the microbial communities were strongly
influenced by plant identity, which is in line with numerous studies
which reported that plants select their microbial communities (Chaparro,
Badri \& Vivanco, 2014; Reinhold-Hurek et al., 2015).\\
\hspace*{0.333em}\\
We also aimed to identify candidate taxa positively correlated with
increased plant productivity in response to ANE application. In fungi,
one cluster of ASV taxonomically assigned to \emph{Mortierella} (soil
saprotrophs in the phylum Zygomycota) was positively correlated to
productivity in both tomato and pepper roots. In their study, Chung et
al. (2007) showed how higher plant species richness and increase in
productivity led to greater microbial biomass and greater number of
saprophytic and arbuscular mycorrhizal fungi. Perhaps, this can be
explained by the fact that microbial communities experienced greater
substrate availability, potentially increasing their activity, and the
activity of saprophytic fungi feeding on organic matter.\\
\hspace*{0.333em}\\
In addition, we identified several fungal ASV in tomato soil and
pepper-root linked to increases in productivity. A number of putative
plant pathogenic fungi were also identified such as \emph{Fusarium sp.},
\emph{Microdochium colombiense} or \emph{Setophoma terrestris} (Figure
S2). In bacteria roots samples, a diverse number of ASV were positively
impacted by fertilization (Figure S3). The specific role of those taxa
on crop productivity will need further investigations.\\
\hspace*{0.333em}\\
It is now well established that seaweed extracts have a positive effect
on agricultural plant productivity. Concurrently, DNA barcoding permits
a more comprehensive understanding of the diversity and ecology of
microbial organisms and how they interact. In fact, plants and microbes
should likely be redefined as \emph{holobionts}, an assemblage of
different species that form an ecological unit (Margulis \& Fester,
1991). In this study, we showed that the addition of ANE increased plant
productivity. It also increased, by a small, but significant margin, the
fungal and bacterial (only in the rhizosphere) biodiversity and changed
the microbial community structure in the roots and in the rhizosphere of
the plants. Finally, we identified bacterial and fungal taxa, especially
saprotroph, that were positivity associated with plant productivity.
Further studies, for example using inoculum of microbial species linked
to increases in productivity and the presence of liquid seaweed extract,
may help to identify a causative link between extracts, microbes and
productivity.\\
\hspace*{0.333em}

\section{ACKNOWLEDGMENTS}\label{acknowledgments}

We thank Mengxuan Kong for technical assistance in setting up the
greenhouse experiment and measuring productivity; Mulan Dai for
performing preliminary microbiome analysis and Simon Morvan for
discussion about bioinformatics analyses and seaweed extracts. Research
funding was provided by the Quebec Centre for Biodiversity Science
(FRQNT) to SR, and NSERC to MH. In-kind contributions were provided by
Acadian Seaplants Ltd. \newpage  

\section*{REFERENCES}\label{references}
\addcontentsline{toc}{section}{REFERENCES}

\hypertarget{refs}{}
\hypertarget{ref-alam2013effect}{}
Alam MZ., Braun G., Norrie J., Hodges DM. 2013. Effect of ascophyllum
extract application on plant growth, fruit yield and soil microbial
communities of strawberry. \emph{Canadian Journal of Plant Science}
93:23--36.

\hypertarget{ref-Alam2014}{}
Alam MZ., Braun G., Norrie J., Hodges DM. 2014. Ascophyllum extract
application can promote plant growth and root yield in carrot associated
with increased root-zone soil microbial activity. \emph{Canadian Journal
of Plant Science} 94:337--348. DOI:
\href{https://doi.org/10.4141/cjps2013-135}{10.4141/cjps2013-135}.

\hypertarget{ref-allen2001tasco}{}
Allen V., Pond K., Saker K., Fontenot J., Bagley C., Ivy R., Evans R.,
Schmidt R., Fike J., Zhang X., others. 2001. Tasco: Influence of a brown
seaweed on antioxidants in forages and livestock---A review 1.
\emph{Journal of Animal Science} 79:E21--E31.

\hypertarget{ref-altschul1997gapped}{}
Altschul SF., Madden TL., Schäffer AA., Zhang J., Zhang Z., Miller W.,
Lipman DJ. 1997. Gapped blast and psi-blast: A new generation of protein
database search programs. \emph{Nucleic acids research} 25:3389--3402.

\hypertarget{ref-anderson2001new}{}
Anderson MJ. 2001. A new method for non-parametric multivariate analysis
of variance. \emph{Austral ecology} 26:32--46.

\hypertarget{ref-anderson1999empirical}{}
Anderson MJ., Legendre P. 1999. An empirical comparison of permutation
methods for tests of partial regression coefficients in a linear model.
\emph{Journal of statistical computation and simulation} 62:271--303.

\hypertarget{ref-ayad1998effect}{}
Ayad J. 1998. The effect of seaweed extract (ascophyllum nodosum) on
antioxidant activities and drought tolerance of tall fescue (festuca
arundinacea schreb). \emph{Ph D Thesis, Texas Tech University}.

\hypertarget{ref-ayad1997effect}{}
Ayad J., Mahan J., Allen V., Brown C. 1997. Effect of seaweed extract
and the endophyte in tall fescue on superoxide dismutase, glutathione
reductase and ascorbate peroxidase under varying levels of moisture
stress. In: \emph{American forage and grassland council conference
proceedings}.

\hypertarget{ref-Battacharyya2015}{}
Battacharyya D., Babgohari MZ., Rathor P., Prithiviraj B. 2015. Seaweed
extracts as biostimulants in horticulture. \emph{Scientia Horticulturae}
196:39--48. DOI:
\href{https://doi.org/10.1016/j.scienta.2015.09.012}{10.1016/j.scienta.2015.09.012}.

\hypertarget{ref-silva}{}
Callahan B. 2018. Silva for dada2: Silva taxonomic training data
formatted for dada2 (silva version 132). \emph{Zenodo}. DOI:
\href{https://doi.org/10.5281/zenodo.1172783}{10.5281/zenodo.1172783}.

\hypertarget{ref-callahan2016dada2}{}
Callahan BJ., McMurdie PJ., Rosen MJ., Han AW., Johnson AJA., Holmes SP.
2016. DADA2: High-resolution sample inference from illumina amplicon
data. \emph{Nature methods} 13:581.

\hypertarget{ref-caporaso2010qiime}{}
Caporaso JG., Kuczynski J., Stombaugh J., Bittinger K., Bushman FD.,
Costello EK., Fierer N., Pena AG., Goodrich JK., Gordon JI., others.
2010. QIIME allows analysis of high-throughput community sequencing
data. \emph{Nature methods} 7:335.

\hypertarget{ref-chaparro2014rhizosphere}{}
Chaparro JM., Badri DV., Vivanco JM. 2014. Rhizosphere microbiome
assemblage is affected by plant development. \emph{The ISME journal}
8:790.

\hypertarget{ref-chung2007plant}{}
Chung H., Zak DR., Reich PB., Ellsworth DS. 2007. Plant species
richness, elevated co2, and atmospheric nitrogen deposition alter soil
microbial community composition and function. \emph{Global Change
Biology} 13:980--989.

\hypertarget{ref-UNITE2017}{}
Community U. 2018.UNITE general fasta release. version 01.12.2017.
\emph{Available at}
\emph{\url{https://files.plutof.ut.ee/doi/C8/E4/C8E4A8E6A7C4C00EACE3499C51E550744A259A98F8FE25993B1C7B9E7D2170B2.zip}}

\hypertarget{ref-Craigie2010}{}
Craigie JS. 2010. Seaweed extract stimuli in plant science and
agriculture. \emph{Journal of Applied Phycology} 23:371--393. DOI:
\href{https://doi.org/10.1007/s10811-010-9560-4}{10.1007/s10811-010-9560-4}.

\hypertarget{ref-craigie2011seaweed}{}
Craigie JS. 2011. Seaweed extract stimuli in plant science and
agriculture. \emph{Journal of Applied Phycology} 23:371--393.

\hypertarget{ref-dhargalkar2005seaweed}{}
Dhargalkar V., Pereira N. 2005. Seaweed: Promising plant of the
millennium.

\hypertarget{ref-fisher1943relation}{}
Fisher RA., Corbet AS., Williams CB. 1943. The relation between the
number of species and the number of individuals in a random sample of an
animal population. \emph{The Journal of Animal Ecology}:42--58.

\hypertarget{ref-duJardin2015}{}
Jardin P du. 2015. Plant biostimulants: Definition, concept, main
categories and regulation. \emph{Scientia Horticulturae} 196:3--14. DOI:
\href{https://doi.org/10.1016/j.scienta.2015.09.021}{10.1016/j.scienta.2015.09.021}.

\hypertarget{ref-Jayaraj2015sustainable}{}
Jayaraj J., Ali N. 2015. Use of seaweed extracts for disease management
of vegetable crops. In: Ganesan S, Vadivel K, Jayaraman J eds.
\emph{Sustainable crop disease management using natural products}. CAB
International, 160--183.

\hypertarget{ref-Jayaraj2008}{}
Jayaraj J., Wan A., Rahman M., Punja Z. 2008. Seaweed extract reduces
foliar fungal diseases on carrot. \emph{Crop Protection} 27:1360--1366.
DOI:
\href{https://doi.org/10.1016/j.cropro.2008.05.005}{10.1016/j.cropro.2008.05.005}.

\hypertarget{ref-Jayaraman2010}{}
Jayaraman J., Norrie J., Punja ZK. 2010. Commercial extract from the
brown seaweed ascophyllum nodosum reduces fungal diseases in greenhouse
cucumber. \emph{Journal of Applied Phycology} 23:353--361. DOI:
\href{https://doi.org/10.1007/s10811-010-9547-1}{10.1007/s10811-010-9547-1}.

\hypertarget{ref-jithesh2012analysis}{}
Jithesh MN., Wally OS., Manfield I., Critchley AT., Hiltz D.,
Prithiviraj B. 2012. Analysis of seaweed extract-induced transcriptome
leads to identification of a negative regulator of salt tolerance in
arabidopsis. \emph{HortScience} 47:704--709.

\hypertarget{ref-jukes1969evolution}{}
Jukes T., Cantor C. 1969. \emph{Evolution of protein molecules, pp.
21--132 in mammalian protein metabolism, edited by munro hn}. Academic
Press, New York.

\hypertarget{ref-khan2009seaweed}{}
Khan W., Rayirath UP., Subramanian S., Jithesh MN., Rayorath P., Hodges
DM., Critchley AT., Craigie JS., Norrie J., Prithiviraj B. 2009. Seaweed
extracts as biostimulants of plant growth and development. \emph{Journal
of Plant Growth Regulation} 28:386--399.

\hypertarget{ref-klindworth2013evaluation}{}
Klindworth A., Pruesse E., Schweer T., Peplies J., Quast C., Horn M.,
Glöckner FO. 2013. Evaluation of general 16S ribosomal rna gene pcr
primers for classical and next-generation sequencing-based diversity
studies. \emph{Nucleic acids research} 41:e1--e1.

\hypertarget{ref-lizzi1998seaweed}{}
Lizzi Y., Coulomb C., Polian C., Coulomb P., Coulomb P. 1998. Seaweed
and mildew: What does the future hold? \emph{Phytoma La Defense des
Vegetaux (France)}.

\hypertarget{ref-margulis1991symbiosis}{}
Margulis L., Fester R. 1991. \emph{Symbiosis as a source of evolutionary
innovation: Speciation and morphogenesis}. Mit Press.

\hypertarget{ref-milton1952improvements}{}
Milton R. 1952. Improvements in or relating to horticultural and
agricultural fertilizers. \emph{British Patent} 664989.

\hypertarget{ref-Newman2013}{}
Newman M-A., Sundelin T., Nielsen JT., Erbs G. 2013. MAMP
(microbe-associated molecular pattern) triggered immunity in plants.
\emph{Frontiers in Plant Science} 4. DOI:
\href{https://doi.org/10.3389/fpls.2013.00139}{10.3389/fpls.2013.00139}.

\hypertarget{ref-oksanen2013package}{}
Oksanen J., Blanchet FG., Kindt R., Legendre P., Minchin PR., O'hara R.,
Simpson GL., Solymos P., Stevens MHH., Wagner H., others. 2013. Vegan:
Community ecology package. r package version 1.17.2. \emph{R software}.

\hypertarget{ref-paradis2004ape}{}
Paradis E., Claude J., Strimmer K. 2004. APE: Analyses of phylogenetics
and evolution in r language. \emph{Bioinformatics} 20:289--290.

\hypertarget{ref-pinheiro2017nlme}{}
Pinheiro J., Bates D., DebRoy S., Sarkar D., Team RC. 2017. Nlme: Linear
and nonlinear mixedeffects models. r package version 3.1-128. \emph{R
software}.

\hypertarget{ref-team2018r}{}
R Core Team. 2018. R: A language and environment for statistical
computing.

\hypertarget{ref-reinhold2015roots}{}
Reinhold-Hurek B., Bünger W., Burbano CS., Sabale M., Hurek T. 2015.
Roots shaping their microbiome: Global hotspots for microbial activity.
\emph{Annual review of phytopathology} 53:403--424.

\hypertarget{ref-schliep2010phangorn}{}
Schliep KP. 2010. Phangorn: Phylogenetic analysis in r.
\emph{Bioinformatics} 27:592--593.

\hypertarget{ref-schloss2009introducing}{}
Schloss PD., Westcott SL., Ryabin T., Hall JR., Hartmann M., Hollister
EB., Lesniewski RA., Oakley BB., Parks DH., Robinson CJ., others. 2009.
Introducing mothur: Open-source, platform-independent,
community-supported software for describing and comparing microbial
communities. \emph{Applied and environmental microbiology}
75:7537--7541.

\hypertarget{ref-schmidt1997influence}{}
Schmidt R., Zhang X. 1997. Influence of seaweed on growth and stress
tolerance of grasses. In: \emph{American forage and grassland council
conference proceedings}. Ft. Worth, TX, 158--162.

\hypertarget{ref-spann2011applications}{}
Spann TM., Little HA. 2011. Applications of a commercial extract of the
brown seaweed ascophyllum nodosum increases drought tolerance in
container-grown `hamlin'sweet orange nursery trees. \emph{HortScience}
46:577--582.

\hypertarget{ref-toju2012high}{}
Toju H., Tanabe AS., Yamamoto S., Sato H. 2012. High-coverage its
primers for the dna-based identification of ascomycetes and
basidiomycetes in environmental samples. \emph{PloS one} 7:e40863.

\hypertarget{ref-wally2013regulation}{}
Wally OS., Critchley AT., Hiltz D., Craigie JS., Han X., Zaharia LI.,
Abrams SR., Prithiviraj B. 2013. Regulation of phytohormone biosynthesis
and accumulation in arabidopsis following treatment with commercial
extract from the marine macroalga ascophyllum nodosum. \emph{Journal of
plant growth regulation} 32:324--339.

\hypertarget{ref-wang2007naive}{}
Wang Q., Garrity GM., Tiedje JM., Cole JR. 2007. Naive bayesian
classifier for rapid assignment of rRNA sequences into the new bacterial
taxonomy. \emph{Applied and environmental microbiology} 73:5261--5267.

\hypertarget{ref-wickham2016ggplot2}{}
Wickham H. 2016. \emph{Ggplot2: Elegant graphics for data analysis}.
Springer.

\hypertarget{ref-wickham2015dplyr}{}
Wickham H., Francois R., Henry L., Müller K. 2015. Dplyr: A grammar of
data manipulation. \emph{R package version 0.4} 3.

\hypertarget{ref-wright2016using}{}
Wright ES. 2016. Using decipher v2.0 to analyze big biological sequence
data in r. \emph{R Journal} 8:352--359.




\newpage
\singlespacing 
\end{document}
